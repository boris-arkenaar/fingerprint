\section{Introduction}

The main goal is to get control over the fingerprint of the browser.
Having that control will allow us
to study the effect of different fingerprints
on the act of fingerprinting.
The act of fingerprinting uses the fingerprint
to track a specific user
and possibly serve specific content to that user
based on the tracking data.
When we have control over the fingerprint the web server \emph{sees}
when we visit a website,
we can study if and how the content of that website changes
based on differences in the fingerprint we \emph{serve}.

\section{Controlling the fingerprint}

Many articles have been written on changing the fingerprint in a random fashion
to prevent the act of fingerprinting from tracking someone.
We need not simply change the original fingerprint,
but be able to specify certain characteristics
the generated fingerprint should have.
All the characteristics not explicitly provided
should still be generated in a random fashion,
but the resulting fingerprint
should be a plausible combination of characteristics.

Baumann et al. use correct fingerprints to replace the original one.
\citep{baumann2016disguised}
They do this by storing the fingerprints
of every user of their created browser
on a server and using these \emph{real} fingerprints to substitute others.
With FP-Block fingerprints are really generated.
\citep{torres2015fp-block:}
By making use of Markov chains they make sure
that each generated fingerprint is a plausible one.
