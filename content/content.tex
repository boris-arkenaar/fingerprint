\section{Introduction}

The main goal is to get control over the fingerprint of the browser.
Having that control will allow us
to study the effect of different fingerprints
on the act of fingerprinting.
The act of fingerprinting uses the fingerprint
to track a specific user
and possibly serve specific content to that user
based on the tracking data.
When we have control over the fingerprint the web server \textit{sees}
when we visit a website,
we can study if and how the content of that website changes
based on differences in the fingerprint we \textit{serve}.

\section{Controlling the fingerprint}

Many articles have been written on changing the fingerprint in a random fashion
to prevent the act of fingerprinting from tracking someone.
We need not simply change the original fingerprint,
but be able to specify certain characteristics
the generated fingerprint should have.
All the characteristics not explicitly provided
should still be generated in a random fashion,
but the resulting fingerprint
should be a plausible combination of characteristics.

Baumann et al. use correct fingerprints to replace the original one.
\cite{baumann2016disguised}
They do this by storing the fingerprints
of every user of their created browser
on a server and using these \textit{real} fingerprints to substitute others.
With FP-Block fingerprints are really generated.
\cite{torres2015fp-block:}
By making use of Markov chains they make sure
that each generated fingerprint is a plausible one.

\section{Possible studies}

\subsection{Price discrimination airline tickets}

It would be interesting to see if airlines use price discrimination
on their tickets sold online based on the fingerprint of the visitor.
When that is the case we would like to know
how the price differs and based on which values of a fingerprint.
A similar study has been performed recently however,
with the conclusion that no price discrimination was being applied
by any of the airlines included in the study.
\cite{vissers2014crying}

\subsection{Malvertisement}

Would malvertisement be mainly directed toward
fingerprints containing certain characteristics
indicating they would be more vulnerable to an attack?
It already seems that attackers have a preference
for the Internet Explorer browser
when targeting malvertisement.
\cite{li2012knowing}

\subsection{A/B testing}
Websites make use of A/B testing to figure out the impact
of two different approaches to a change
in their design, usability of functionality.
It would be interesting to see if some A/B test
is being based on the fingerprint of the visitor.
The fingerprint might be used to determine
if the visitor at hand will be included in the A/B test
or that he will be presented the \textit{normal} version of the website.
In case the visitor is included in the test
the fingerprint might be used to determine which version of the change,
the A or the B version, will be presented.
Although that might skew the results of the A/B test,
since it is much more than simply an A/B test that way.

\subsection{Fingerprint generator}
A more theoretical approach is to make a study on what a fingerprint exists of.
What are all the possible properties and their values,
What are plausible combinations of these values
and how can these be generated?
Important as well is to find out the best way
to keep this generator up to date.
Since browsers, the web and fingerprinters keep evolving
it must be possible to add, remove and update
the fingerprint properties, values and their combinations,
in order for this generator to keep being useful.

\subsection{Whatsapp/Facebook interaction on advertisement}
Whatsapp supposedly encrypts your chats end to end
and as such should no be able to read along.
After clicking a link in a Whatsapp chat, however,
Advertisements about that same link pop up on Facebook.
Is Facebook still able to read along with your chats
and serve you ads based on you chat history?
Or is Facebook \textit{simply} making use of your fingerprint
when you visit the link from the chat in your browser,
and serve you ads based on your browsing history?

\section{Related work}

\subsection{Fingerprinting}

Current technological possible ways of fingerprinting
are being discussed in \cite{nikiforakis2014workings}.
They highlight the possibilities of current fingerprinters,
of plug-ins that prevent you from being tracked
and of their shortcomings.

\subsection{FPDetect}

Instead of defending against known fingerprinters
and fingerprinting techniques,
\cite{acar2013fpdetective} developed a framework
for detecting active fingerprinters.
They found out
that much more fingerprinters are active then previously expected.

\subsection{Basic fingerprinting}

This article discusses basic fingerprinting
by making use of only IP address, fonts, timezone and screen resolution.
\cite{boda2011user}

\section{Tracking preferences}

Mostly directed towards the perception by users of being tracked
and their preference therein.
\cite{melicher2015(do}
They also discuss current tools to take control over you being tracked
and their limitations.
Concluding with a suggestion on how to determine
how much a user would like to be tracked in a certain situation
and how to enforce this preference.
